\chapter{Working environment}
In section \ref{sec:vmclass} we introduced a classification of Virtual Machine systems.

The work presented in this thesis is restricted to same-ISA System Virtual Machines, where the Virtual Machine Monitor is a type 2 VMM.
In other words, we will deal with VMs that able to run an arbitrary OS compiled for the host ISA. The guest OS can in turn provide an 
execution environment for many user application.
Since the VMM is of type 2, it is implemented as a regular process in the host OS, and can make use of all the OS service.
We can therefore access the physical resource without requiring administrator previleges.

We also restrict our work to VMMs that make use of hardware-based virtualization, because the optimizations we will intrduce are
particularly effective in limiting the amount of VM switches between the host world and the guest world. Since these VM switches
are very expensive with hardware virtualization, the performance gain is significant.

While the assumptions made may appear restrictive, they are not at all. The class of VMMs that we consider is extremely 
common in the world of computing. They are used in datacenters and IT departments for server consolidation, 
application isolation, or to provide users/developers with zero-setup computing environments and other application in which it's
not important that the computing environment supported the VM has a different ISA from the host ISA.

\vspace{0.5cm}

Several VMM software belonging to the considered class are available. QEMU, VirtualBox, VMWare, Parallels, Windows
Hyper-V or Windows VirtualPC are among the most common examples of this kind of VMMs. These software tools are extremely widespread
and for this reason performance optimizations in these area are certainly useful.

This said, we have chosen the QEMU-KVM Virtual Machine Monitor for implementation and tests, although our optimizations are potentially
applied to VMM of the same class.

\vspace{0.5cm}

Since our optimizations concern network performance, we had to choose a network device to work with. The \emph{e1000} device
was chosen, since it is emulated by the vast majority of VMMs.


\section{QEMU architecture}
QEMU is an open source system emulator, and can actually emulate architecture...