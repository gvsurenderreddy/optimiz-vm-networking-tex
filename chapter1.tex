\chapter{Introduction}

Standard computer systems are hierarchically organized as a three
layers stack. The lowest layer is the bare hardware, the middle one is the operating system and the 
upper layer contains the application software.
Two neighbour layers can communicate through a well defined \emph{interface}, so that each layer can ignore how the
lower layers are actually implemented. In this way, the interface provides an \emph{abstraction} of the underlying software/hardware
systems.

This (relatively) simple architecture has proven to be very effective in delivering the IT services required
by companies, individual users and other organizations.

Neverthless, more complex computer systems organizations have been devised and used, in order to overcome some limitations
of the standard computer system model. These computing environments are are known as \emph{Virtual Machines} (VMs) systems.
By itself, the term \emph{Virtual Machine} can have several meanings, so when using it is important to point out what we are
addressing.


\section{Virtual Machines}



%\begin{enumerate}
%	\item 24 \emph{tiles} (ossia mattonelle) che compongono il cluster: ogni tiles contiene due core IA
%	\item Una rete mesh composta da 24 router con picco di banda sul taglio pari a 256 GB/s
%	\item 4 DDR3 memory controllers integrati
%	\item supporto hardware per lo scambio dei messaggi
%\end{enumerate}

%\begin{figure}[bt]
%\centering
%\includegraphics[scale = 0.45]{SCCBlockDiagram.png}
%\caption{ Schema a blocchi dell'SCC }
%\label{fig:SCC}
%\end{figure}
