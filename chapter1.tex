\chapter{Introduction}

Standard computer systems are hierarchically organized as a three
layers stack. The lowest layer is the bare hardware, the middle one is the operating system and the 
upper layer contains the application software.
Two neighbour layers can communicate through a well-defined \emph{interface}, so that each layer can ignore how the
lower layers are actually implemented. In this way, the interface provides an \emph{abstraction} of the underlying software/hardware
resources.

This (relatively) simple architecture has proven to be very effective in delivering the IT services required
by companies, individual users and other organizations.

\vspace{0.5cm}

Neverthless, more complex computer systems organizations have been devised and used, in order to overcome some limitations
of the standard computer system model. These computing environments are are known as \emph{Virtual Machines} (VMs) systems.
By itself, the term \emph{Virtual Machine} can have several meanings, so when using it is important to point out what we are
addressing.
In general \emph{virtualization} provides a way of increasing the \emph{flexibility} of real hardware/software resources. When a 
resource  is virtualized, it appears to be a different virtual resource or even a \emph{set} of different virtual resources.

As an example, a single IA32 processor can be virtualized in a way that emulates more PowerPC processors. Once this is done, you
can build more standard 3-layer computer systems on top of each emulated PowerPC (virtual) processor, using unmodified OS and 
applications designed to be used on the PowerPC architecture.

In the VMs terminology, each virtualized system is called \emph{guest}, whereas the system providing the virtualization support
is called \emph{host}.
You can virtualize all the resources you wants: disks, network devices, memories and other peripherals.

\vspace{0.5cm}

Generally speaking VMs allow to build computer systems with more abstraction levels than the standard model have. This has important
advantages:
\begin{itemize}
  \item In terms of \emph{flexibility}, using VMs you can easily run programs compiled for a given Instruction Set Architecture (ISA) 
	and a given Operating System (OS) on top of a computer system that has a different ISA and/or a different OS. Using a standard
	system you would be bound to the ISA of your processor and the operating system installed on your machine.

  \item In terms of \emph{protection}, VMs can provide multiple isolated execution environments running on the same phisical machine.
	This allows to execute different applications in different VMs (each VM can have its own OS), so that if an application has a
	security hole, an attacker cannot use the hole to do malicious attacks to an applications running on a different VM. 
	This scenary is still possible when applications are run in the same OS.
  \item In terms of resources usage, VMs can help to reduce hardware costs and power consumption, since they 
\end{itemize}


\section{Virtual Machines}



%\begin{enumerate}
%	\item 24 \emph{tiles} (ossia mattonelle) che compongono il cluster: ogni tiles contiene due core IA
%	\item Una rete mesh composta da 24 router con picco di banda sul taglio pari a 256 GB/s
%	\item 4 DDR3 memory controllers integrati
%	\item supporto hardware per lo scambio dei messaggi
%\end{enumerate}

%\begin{figure}[bt]
%\centering
%\includegraphics[scale = 0.45]{SCCBlockDiagram.png}
%\caption{ Schema a blocchi dell'SCC }
%\label{fig:SCC}
%\end{figure}
