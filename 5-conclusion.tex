\chapter*{Conclusion and future work}
In this work we have shown how is possible to boost the packet-rate performance of an e1000 network adapter in a Virtual
Machine environment.
First of all we analyzed the existing implementation and we pointed out its problems and bottlenecks.
Then we proposed a patch to the QEMU e1000 frontend (the \emph{moderation} patch) and a patch to the Linux e1000 driver (the \emph{batching}
patch) that ecan be used, independently on each other, to improve performances. Both patches are very simple and don't extend the e1000
interface specification.
After that we analyzed the Virtio I/O paravirtualization framework and provided a small extension to the e1000 interface that allows to remove
some inefficiency by porting the I/O paravirtualization concepts to the e1000 platform.

\vspace{0.5cm}

Comparing the TX and RX performances of the existing implementation with our best results (e1000 paravirtualization) we have obtained
a $8.9 \times$ speed-up on the TX path and a $22 \times$ speed-up on the RX side.
This packet rate performance is comparable or superior to the Virtio performance and is obtained with small modification to the e1000
driver and device emulator.

\vspace{0.5cm}

More work can be done to achieve further improvements with the e1000 platform. In particular, three aspects can be optimized:
\begin{itemize}
  \item Packet-rate: Although this parameter is the optimization objective of these thesis, we can further improve performance removing
	some packet copy. For example, the current e1000 device emulation copies a TX packet from the guest memory to a local buffer: this
	copy is essentially useless and it can be avoided with some memory mapping efforts.
	
  \item Latency: Interrupt moderation worsens latency, so we should implement some euristics aimed at bypass moderation interrupt delay
	when the network traffic requires low latency more than high packet rate (e.g. HTTP transactions).
	
  \item TXP throughput: The key for performance here is big packets, more than high packet rate. The bigger packets you can make, the
	higher throughput you can get. TSO/GSO features must be used, like Virtio does.
\end{itemize}
